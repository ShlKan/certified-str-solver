\section{Motivating Example}
\label{sec:motivation}

In this section, we recall a simple XSS example (see Listing
\ref{lst-web-security}) from the CACM article 
\cite{Kern}, which epitomizes the indispensability of the replace-all operator 
and string transductions, as well as other string operators like concatenation.
In particular, string transductions are so common in web applications that they
sometimes even happen silently (see \texttt{innerHTML} below). A more detailed
account on this can be found in \cite{systematic-transduction}.
%Web applications frequently perform complex string manipulations that must handle untrusted user input securely.
%XSS (Cross-Site Scripting) attacks are among top 3 security vulnerabilities in web applications according to OWASP~\cite{owasp-top10}.
%We present a concrete example from \cite{LB16} demonstrating how mutation XSS vulnerabilities can arise from improper string sanitization, illustrating the important role of SFTs
%can play in the verification.
%Listing \ref{lst-web-security} shows a code snippet that is vulnerable to mutation XSS attacks.









% \subsection{Sanitizer}

% An important application of SFTs lies in string sanitization
% for web applications. Sanitizers are crucial for preventing security vulnerabilities such as cross-site scripting (XSS) attacks by ensuring that user inputs are properly encoded before rendering in web browsers. Listing \ref{lst-html-encode} presents a C\# code snippet implementing an HTML encoding function—a fundamental sanitization operation. This function iterates through each character in the input string and encodes characters that are unsafe for HTML rendering.



% \begin{lstlisting}[language={[Sharp]C}, caption={C\# Code for AntiXSS.EncodeHtml version 2.0.}, label={lst-html-encode}, float=htbp]
% static string EncodeHtml(string t)
% {
%   if (t == null) { return null; }
%   if (t.Length == 0) {
%     return string.Empty;
%   }
%   StringBuilder builder = new
%     StringBuilder("", t.Length * 2);
%   foreach (char c in t) {
%     if ((((c > ''') && (c < '{')) |$\mid\mid$|
%          ((c > '@') && (c < '['))) |$\mid\mid$|
%          (((c == ' ') || ((c > '/') && (c < ':'))) |$\mid\mid$|
%          (((c == '.') || (c == ',')) |$\mid\mid$|
%          ((c == '-') || (c == '_'))))) {
%           builder.Append(c);
%       }
%       else {
%           builder.Append("&#" +
%           ((int) c).ToString() + ";");
%       }
%   }
%   return builder.ToString();
% }
% \end{lstlisting}

% Figure \ref{fig:html-sanitizer-sft} illustrates the SFT representation of the HTML encoding sanitizer function. The transducer accepts any input string and produces an output where safe characters are preserved using the identity function $\texttt{id}$, while unsafe characters are encoded as HTML entities. The set of safe characters is defined as $\texttt{safe\_chars} = [\texttt{' '}, \texttt{','}, \texttt{'.'}, \texttt{'-'}, \texttt{'\_'}] \cup [\texttt{'0'}-\texttt{'9'}] \cup [\texttt{'A'}-\texttt{'Z'}] \cup [\texttt{'a'}-\texttt{'z'}]$. Unsafe characters are transformed using the encoding function $\texttt{encode}(c) = \texttt{``\&\#''} + \texttt{toString}(\texttt{ord}(c)) + \texttt{``\;''}$. When the alphabet is set to Unicode $[\mathtt{0x0000}, \mathtt{0x10FFFF}]$, the complement $\neg\texttt{safe\_chars}$ represents all Unicode characters outside the safe set.
% \begin{figure}[htbp]
% \centering
% \begin{tikzpicture}[shorten >=1pt, node distance=2.5cm, on grid, auto]
%   % States
%   \node[state, initial, accepting] (q0) {$q_0$};
%   \node[state] (q1) [right=3.5cm of q0] {$q_1$};
%   \node[state] (q2) [below=of q0] {$q_2$};
  
%   \path[->] 
%     % Transition for safe characters
%     (q0) edge[bend left=20] node[above,midway=1] {$\texttt{safe\_chars}, \texttt{id}$} (q1)
%     % Transition for unsafe characters  
%     (q0) edge[bend right=20] node[left,midway] {$\neg\texttt{safe\_chars}, \texttt{encode}$} (q2)
%     % Epsilon transitions back to q0
%     (q1) edge[bend left=30] node[below,midway] {$\varepsilon, \lambda\varepsilon$} (q0)
%     (q2) edge[bend right=20] node[right,midway] {$\varepsilon, \lambda\varepsilon$} (q0);
% \end{tikzpicture}
% \caption{SFT representation of the HTML encoding sanitizer function.}
% \label{fig:html-sanitizer-sft}
% \end{figure}

% However, incorrectly using sanitizers can still lead to XSS vulnerabilities.
% The following is an example of incorrectly using santizer that leads to XSS vulnerability.

\begin{lstlisting}[language={}, caption={Mutation XSS attack code example},captionpos=b, label={lst-web-security}, float=htbp]
var x = goog.string.htmlEscape(cat);
var y = goog.string.escapeString(x);
catElem.innerHTML = '<button onclick=
"createCatList(\'' + y + '\')">' + x + '</button>';
\end{lstlisting}

The code assigns an HTML markup for a button to the DOM element \texttt{catElem}. This button creates a category named \texttt{cat}, whose value is provided by an untrusted third party. To mitigate potential security risks, the code attempts to sanitize the value of \texttt{cat} using the Closure Library~\cite{google-closure-library} string functions \texttt{htmlEscape} and \texttt{escapeString} (the latter implements JavaScript escaping). 

When the input value \texttt{Flora \& Fauna} is provided for \texttt{cat}, the resulting HTML markup is as intended:
\begin{verbatim}
<button onclick="createCatList('Flora \&amp; Fauna
')"> Flora \&amp; Fauna</button>
\end{verbatim}
On the other hand, inputting the value ');alert(1);// to cat,
results in the HTML markup:
\begin{verbatim}
  <button onclick="createCatList('&#39;);alert(1);
  //')"> &#39;);alert(1);//')</button>
\end{verbatim}

When this markup is inserted into the DOM via \texttt{innerHTML}, 
the resulting string appears as \verb|"createCatList('');alert| (\verb|1);..."|. It enables the attacker's script \texttt{alert(1)} to execute immediately after \texttt{createCatList}. 
This subtle XSS vulnerability does not stem from the absence of escaping functions, but rather from applying them in the incorrect order.

Both \texttt{htmlEscape} and \texttt{escapeString} can be naturally modeled as SFTs, since they transform input strings into output strings. The above code snippet can be represented by the following string constraints (where \texttt{w1},
\texttt{w2}, and \texttt{w3} are constant strings):

\begin{verbatim}
  1. x = SFT_htmlEscape(cat)
  2. y = SFT_escapeString(x)
  3. z = w1 + y + w2 + x + w3;
  4. catElem.innerHTML = SFT_innerHTML(z)
  5. catElem.innerHTML matches R_unsafe
\end{verbatim}
\texttt{SFT\_innerHTML} is a transducer implementing the browser transductions upon innerHTML
assignments. \texttt{R\_unsafe} is a regular expression defining unsafe HTML patterns that can lead to XSS vulnerabilities.

In the remainder of this paper, we formalize SFTs and demonstrate how they can be used to verify the correctness of sanitizer functions. We also show how to integrate SFT-based reasoning into existing string solvers, enabling support not only for basic regular and concatenation constraints, but also for more complex string verification problems involving replacement operations.





% \subsection{Modeling the Replacement Operation}

% The second motivation is the replacement operation for SMT string 
% solvers.



% The string replacement, denoted as
% $\texttt{replace}(\texttt{str}, \texttt{pattern}, \newline\texttt{replacement}, \texttt{flag})$, is a fundamental string transformation accepting four parameters:
% \begin{itemize}
%   \item $\texttt{str}$: The input string to be transformed.
%   \item $\texttt{pattern}$: A regular expression defining the matching criteria.
%   \item $\texttt{replacement}$: The string that replaces matched substrings.
%   \item $\texttt{flag}=s/g$: The replacement mode flag, where $s$ denotes single occurrence replacement and $g$ denotes replace-all operation.
% \end{itemize}

% The operation's semantics is to replace the first occurrence (or all occurrences) of any substring $s'$ in \texttt{str} that matches $\texttt{pattern}$ with \texttt{replacement}.

% To illustrate how SFTs model this operation, consider $\texttt{replace}(s, \texttt{/[0-9]+/}, "\texttt{NUM}", \texttt{flag})$. We construct the SFT representation through three steps:

% \begin{enumerate}
%   \item Construct an SFA recognizing the pattern $\texttt{/[0-9]+/}$ and transform it into an SFT by augmenting each transition with output function "$f = \lambda x.~\texttt{None}$" that produces the empty string.
%   \item Construct an SFA accepting the replacement string "\texttt{NUM}". Since constant strings are specialized regular expressions, we build their SFA representation and convert it to an SFT by adding $\varepsilon$-transitions with output functions that emit the replacement characters sequentially.
%   \item Compose the two SFTs by connecting the accepting state of the pattern-matching SFT to the initial state of the replacement-generating SFT via $\varepsilon$-transitions.
% \end{enumerate}

% Figure \ref{fig-rearranged-automata} presents the complete SFT construction, obtained by composing the pattern-matching SFT generated from \texttt{/[0-9]+/}  with the replacement-generating SFT generated from "\texttt{NUM}", where the function \texttt{f} is defined as "$f = \lambda x.~\texttt{None}$" and the function \texttt{g} is defined as:
% \[
% \texttt{g} = \lambda~i~x.~\texttt{match}~i~\texttt{with}~
% \begin{cases}
% 1 \mapsto [(78, 78)] & \text{(ASCII for 'N')} \\
% 2 \mapsto [(85, 85)] & \text{(ASCII for 'U')} \\
% 3 \mapsto [(77, 77)] & \text{(ASCII for 'M')} \\
% \_ \mapsto \texttt{None}
% \end{cases}
% \]

% \begin{figure}[h] \centering
%   \begin{tikzpicture}[shorten >=1pt, node distance=2cm, auto]
%     % Second automaton from the first figure
%     \node[state, initial] (q0') {$q_0'$}; 
%     \node[state] (q1') [right=of q0'] {$q_1'$}; 
  
%      \path[->] 
%      (q0') edge [bend left] node {\texttt{[0-9]}, $\texttt{f}$} (q1')
%      (q1') edge [loop right] node {\texttt{[0-9]}, $\texttt{f}$} (q1');
  
%     % Second automaton from the second figure
%     \node[state] (q0'') [below=1cm of q1'] {$q_0''$}; 
%     \node[state] (q1'') [right=of q0''] {$q_1''$}; 
%     \node[state, accepting] (q3'') [below=1cm of q0''] {$q_3''$}; 
%     \node[state] (q2'') [right=of q3''] {$q_2''$}; 
  
%     \path[->] 
%     (q0'') edge node {\texttt{None}, \texttt{(g 1)}} (q1'')
%     (q1'') edge [bend left=5] node[left] {\texttt{None}, \texttt{(g 2)}} (q2'')
%     (q2'') edge node {\texttt{None}, \texttt{(g 3)}} (q3'')
%     (q1') edge [bend right=30] node {\texttt{None}, \texttt{f}} (q0'');
%   \end{tikzpicture}
%   \caption{The SFT for the replacement operation \\ $\texttt{replace}(s, \texttt{/[0-9]+/}, \text{``NUM''})$}
%   \label{fig-rearranged-automata}
%   \end{figure}

%   We can extend the above construction to model both singleton and replace-all operations. In this paper, we only illustrate the construction of replace-all in Figure \ref{fig:replace-all}. The SFT $\mathcal{T}_2$ combines pattern matching and replacement using the construction just described. The SFT $\mathcal{T}_1$ matches strings that do not contain any substrings matching the pattern. Two $\varepsilon$-transitions connect these components: one from $\mathcal{T}_1$'s accepting state to $\mathcal{T}_2$'s initial state, and another from $\mathcal{T}_2$'s accepting state back to $\mathcal{T}_1$'s initial state. This cyclic structure enables the replace-all operation by repeatedly applying pattern replacement until no more matches exist in the input string.


%   Having constructed the SFT, we can now compute the forward image of the replacement operation. Consider the string constraint $s' = \texttt{replace}(s,~\texttt{/[0-9]+/},~\text{"}\texttt{NUM}\text{"})$, where we aim to characterize the possible values of $s'$. Let $\mathcal{T}$ denote the constructed SFT modeling the replacement operation, and let $\mathcal{A}$ be the SFA representing the domain of possible values for the input string $s$. The forward image of this replacement operation is given by the product $\mathcal{T} \times \mathcal{A}$, which precisely captures the set of all possible output strings that can be produced by applying the replacement operation to any input string accepted by $\mathcal{A}$.
%   Our string solver uses this forward image to do forward propagation for further string solving.


% \begin{figure}[htbp]
% \centering
% \begin{tikzpicture}[node distance=2cm, state/.style={circle, draw, minimum size=0.8cm}]
%   % First SFT rectangle
%   \node[draw, rectangle, minimum width=5.2cm, minimum height=2.1cm] (sft1-box) {};
%   \node[above=0.1cm of sft1-box.north] {\textbf{SFT} $\mathcal{T}_1$ (Non-matching)};
  
%   % States inside first SFT
%   \node[state, initial] (p0) at ([xshift=-0.8cm]sft1-box.center) {$p_0$};
%   \node[state, accepting] (p1) at ([xshift=1.8cm]sft1-box.center) {$p_1$};
%   \path[->] (p0) edge[above] node {\shortstack{$\neg\texttt{pattern},$ \\ $\lambda x.x$}} (p1);
  
%   % Second SFT rectangle
%   \node[draw, rectangle, minimum width=5.2cm, minimum height=2.1cm, below=1.5cm of sft1-box] (sft2-box) {};
%   \node[below=0.1cm of sft2-box.south] {\textbf{SFT} $\mathcal{T}_2$ (Matching)};
  
%   % States inside second SFT
%   \node[state, initial] (s0) at ([xshift=-0.8cm]sft2-box.center) {$s_0$};
%   \node[state, accepting] (s1) at ([xshift=2cm]sft2-box.center) {$s_1$};
%   \path[->] (s0) edge[above] node {\shortstack{$\texttt{pattern},$ \\ $\texttt{replacement}$}} (s1);
  
%   % Epsilon transition from accepting state of first SFT to initial state of second SFT
%   \path[->] (p1) edge[bend right=30] node[left] {\shortstack{$\varepsilon$, $\lambda\varepsilon$}} (s0);
  
%   % Epsilon transition from accepting state of second SFT to initial state of first SFT
%   \path[->] (s1) edge[bend right=30] node[right] {\shortstack{$\varepsilon$, $\lambda\varepsilon$}} (p0);
  
% \end{tikzpicture}
% \caption{The modeling of replace all operation.}
% \label{fig:replace-all}
% \end{figure}
