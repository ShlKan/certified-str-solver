\section{Related Work}
\label{sec:related-work}

\emph{Symbolic Automata and Transducers.} Symbolic Automata and Transducers \cite{cav/DAntoniV17,VeanesHLMB12Transducer, popl/DAntoniV14, entcs/DAntoniKW18, sofsem/TammV18} represent a significant advancement in automata theory, offering improved efficiency in operations and enhanced expressiveness through algebraic theories that support infinite alphabets. This symbolic framework has been progressively extended to accommodate more complex structures: symbolic tree transducers \cite{ershov/VeanesB11} handle hierarchical data structures, while symbolic pushdown automata \cite{cav/DAntoniA14} manage nested word structures. 
%Recent developments in 2024 have further expanded the scope of symbolic techniques to include B\"{u}chi automata and omega-regular languages \cite{pacmpl/VeanesBEZ25}, enabling verification of infinite-state systems and analysis of non-terminating computations.

\emph{Applications of Symbolic Transducers.} The efficiency, scalability, and
expressive power of symbolic automata and transducers have led to their
widespread adoption in numerous applications, including: constraint solving for program analysis \cite{lpar/VeanesBM10}, security-critical sanitizer analysis for web applications \cite{uss/HooimeijerLMSV11}, runtime verification of system behaviors \cite{osdi/YaseenABCL20}, automated program inversion for software 
transformation \cite{pldi/HuD17}, and string solving
\cite{pacmpl/ChenFHHHKLRW22,CHL+19}. Each application leverages the symbolic 
approach's ability to handle complex
patterns and large alphabets efficiently.

\emph{Formalization of Symbolic Automata and Transducers.} While classical automata theories have been extensively formalized in interactive theorem provers \cite{Tuerk-NFA, Lammich2014TheCA, Peter14, cpp/DoczkalKS13}, with some work on transducer formalization \cite{afp/LochmannFSTS21}, the symbolic variants of automata and transducers remain largely unexplored in formal verification. 
%To our knowledge, CertiStr \cite{cpp/KanLRS22} represents the only existing work on symbolic automata formalization in a proof assistant. Our work advances this frontier by extending the formal treatment to both SFTs and $\varepsilon$SFAs.


\emph{String Solving.} A significant application of our certified transducer
framework is in string constraint solving, a field that has seen intensive
research development over the past decade. While our work provides formal
verification guarantees, there exists a rich ecosystem of non-certified string
solvers, which has in the last few years led to a standardized SMT-LIB 2.6/2.7 
unicode string theory and annual competition. SMT-LIB compatible string 
solvers include Z3 \cite{Z3}, Z3-alpha \cite{z3alpha}, Z3-Noodler 
\cite{noodler,noodler-int,noodler-len, noodler-tool}, Z3str3RE \cite{z3str3re},
cvc5 \cite{cvc5}, Z3str4 \cite{z3str4}, Trau \cite{Z3-trau,trau-tool}, and 
OSTRICH \cite{ostrich,CHL+19,pacmpl/ChenFHHHKLRW22}. 
Recent SMT competition over string theory has revealed
superiority of automata-based methods (based on restrictions of symbolic
automata). 
We envision that our work will benefit the string solving community by 
providing a formal foundation for string solvers development.
%: Kaluza \cite{Berkeley-JavaScript} specializes in JavaScript analysis, CVC5 \cite{cvc5}, Z3-str3 \cite{Z3-str3} builds on the Z3 framework, S3P \cite{DBLP:conf/ccs/TrinhCJ14}, Ostrich \cite{pacmpl/ChenFHHHKLRW22}, and SLOTH \cite{HJLRV18}.
%As more and more 
%Numerous bugs have been uncovered in existing string solvers \cite{DBLP:conf/cav/BlotskyMBZKG18} and SMT solvers \cite{Mansur20} 



\emph{Certified SMT Solvers.} Certification efforts 
for SMT (other than string theories) also exist. For example, the work by Shi et al. \cite{DBLP:conf/cav/ShiFLTWY20}, who developed a certified SMT solver for quantifier-free bit-vector theory, demonstrating the broader applicability of interactive theorem proving in certified SMT solver development. The work by Mansur et al. \cite{verified-verifying} define a formal semantics for SMT-LIB string
to check compatibility between solvers and SMTLIB standard.
Finally, there are also recent works on certified proof checkers for automated 
reasoning and in particular SMT (cf. \cite{cacm-certified}).
